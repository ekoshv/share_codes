\documentclass{article}
\usepackage{lipsum} % for generating filler text
\usepackage{amsmath} % for math equations and symbols
\usepackage{hyperref} % for hyperlinks
\usepackage{tikz}
\usetikzlibrary{shapes,arrows}


\title{Real-Time Model-Based Control of Complex Industrial Systems using Physics Informed Neural Networks and FPGA/ASIC}
\author{Ehsan KhademOlama}
\date{\today}

\begin{document}
	
	\maketitle
	
	\section{Introduction}
	
	In the field of control systems for complex industrial systems, the need for real-time, accurate, and efficient control is paramount. Traditional control methods, while effective, often require extensive system knowledge and are limited in their ability to adapt to complex or changing system dynamics. This paper proposes a novel approach to address these challenges using Physics Informed Neural Networks (PINNs), advanced control methods, and Field Programmable Gate Arrays (FPGAs) or Application-Specific Integrated Circuits (ASICs).
	
	\section{Physics Informed Neural Networks}
	
	PINNs are a type of neural network that incorporate known physical laws, represented by ordinary differential equations (ODEs) or partial differential equations (PDEs), into their structure. By training a PINN on data from the industrial system, we can create a model that not only fits the data but also respects the underlying physics. This leads to more accurate and reliable predictions, which are crucial for effective control. The use of PINNs allows us to leverage the power of machine learning while also incorporating our knowledge of the physical laws governing the system.
	
	\section{FPGA/ASIC Implementation}
	
	Once the PINN is trained, it can be transferred to an FPGA or ASIC. These hardware platforms are known for their high performance and efficiency, making them ideal for real-time applications. By running the PINN on an FPGA or ASIC, we can achieve real-time performance, which is crucial for control applications. The use of FPGA/ASIC technology allows us to leverage the speed and efficiency of hardware acceleration, enabling real-time control even for complex industrial systems.
	
	\section{Advanced Controllers}
	
	In addition to the PINN, we also propose the use of advanced controllers, such as Model Predictive Control (MPC) or Sliding Mode Control (SMC). These controllers use the model to predict the system's future behavior and make control decisions accordingly. Because the model is running in real-time on the FPGA or ASIC, these decisions can be made very quickly, allowing the controller to respond to changes in the system immediately. The use of advanced controllers allows us to leverage the power of predictive and robust control methods, improving the performance and stability of the system.
	
	\section{Applications}
	
	The end result is a real-time model-based controller for the complex industrial system. This controller uses the PINN model to make informed control decisions that respect the underlying physics of the system. Because the model and controller are running on an FPGA or ASIC, these decisions can be made in real-time, allowing the system to respond quickly to changes.
	
	This approach combines the power of machine learning with the efficiency of hardware acceleration to create a highly effective control system for complex industrial systems. It could be applied to a wide range of systems, from petrochemical and chemical applications to industrial polymerization processes, Robotics and Mechatronic Systems, avionics and space technologies and could significantly improve their performance and efficiency.
	
	\section{Challenges and Future Work}
	
	While the proposed approach holds significant promise, it also presents several challenges. These include the computational complexity of the PINN, the availability of training data, the accuracy of the model, the complexity of the control algorithms, and the limitations of the hardware platform. Future work will focus on addressing these challenges and further refining the proposed approach.
	
\end{document}
