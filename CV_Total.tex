\documentclass[fontsize=12pt]{tccv}
\usepackage[english]{babel}
\usepackage{stackengine,graphicx,multicol}
\usepackage{xurl}
\usepackage{wasysym}
\usepackage{marvosym}
\usepackage{textcomp}
\usepackage{ragged2e}
\usepackage{microtype}
\usepackage[T1]{fontenc}
\begin{document}
%\fontfamily{cmr}\selectfont	
	\part{\smash{\belowbaseline[-20pt]{\includegraphics[width=1in]{picture.jpg}}}%
		\hfill Ehsan KhademOlama\hfill\mbox{}}
		
$\null$
\vspace{15mm}
	\personal
	[www.linkedin.com/in/ehsan-khademolama-05149865]
	{2B, Taunusstr, Eschborn, Hessen 65760, Germany}
	{(+49) 176-43165368}
	{ekoshv.igt@gmail.com}
	
	\section{\\Review}
	\begin{justify}
My background in mechatronics and experience in electronic engineering projects, including a strong focus on nonlinear dynamic system identification and optimization, as well as model-based linear/nonlinear controller and observer design, provide me with a skill set that is relevant to the job requirements for algorithm development using inertial sensor data. I am proficient in MATLAB, Python, and embedded C/C++, and have experience in classic signal processing, data ans sensor fusion, and Kalman filters. I also have a good understanding of statistics and mathematical modeling, which are essential for developing and verifying complex algorithms.

Furthermore, I am familiar with industrial sensors, electrical motors, and hydromechanical systems, and have experience in functional testing of motors and electrical conductors, which are relevant to the job. I have designed electronic control board systems, including a 300W brushless DC motor driver board using Altium Designer, which showcases my ability to exploit all the features and characteristics of the hardware to achieve outstanding performance. In addition, I have experience in 3D modeling using SolidWorks and CFD simulation using Ansys and COMSOL, which demonstrates my ability to contribute to technology roadmaps, find new use cases and applications, and generate patents.

As an open-minded team player with excellent communication skills, I am always eager to learn and adapt to new challenges. My experience and skills make me a suitable candidate for positions in algorithm development, sensor fusion, and signal processing in mechatronics, hydraulic systems, control, and robotics. I would be excited to join a team of open-minded people with versatile education, culture, and history who develop sophisticated, cutting-edge algorithms for orientation and motion evaluation.
	\end{justify}
	\vspace{-3mm}
	\section{Work experience}
	
	\begin{eventlist}
		\item{2018 -- present}
		{Intertek Group plc, Arotec Inspection Partner, TÜV Süd, TÜV Nord, TÜV Rheinland, McDermott,ABS Group}
		{Engineering Consultant \& \\ Industrial Quality Control Inspector}
		
		\small \textbf{Experience}:\\
		\small *Industrial sensors and instrumentation fittings equipments and calibration.\\
		\small *Electrical motors and Electro-Hydraulic, Solenoid Actuators.\\
		\small *Inspection of measurement and process tools.\\
		
		\small \textbf{Skills}:\\
		\small *Functional/operational test of motors, electrical conductor's material compliance.\\
		\small *Reading P\&ID diagrams, single/multi line diagrams, and 2/3D schematics of control panels.\\
		\small *Witnessing QC tests as per Quality Control Plan and reference standards.\\
		\small *Witnessing NDT after and before QC tests as per request and review of relative certificates and reports.\\
		
		\small \textbf{Knowledge}:\\
		\small *ASME sections IX, V, B16.5 and B31.1-B31.3.\\
		\small *NEMA MG 1-2016.\\
		\small *IEC 62271-200:2021 (International Electrotechnical Commission Standard for high-voltage switchgear and controlgear).\\
		
		\item{2017 -- 2018}
		{Università degli studi di Bergamo}
		{Research assistant}
		
		\small \textbf{Designing and implementing control systems:}\\
		\small * Model Predictive Controller for Brushless motors based on the State Space Approach.\\
		\small * Velocity Direct Feedback (VDF) for car suspension using sensor fusion based on the Wavelet filters with comparison to Kalman filter methods.
		
		\small \textbf{Working on product and process integration:}\\
		\small * Realization of electric motors for road vehicles in partnership with Brembo S.p.a.\\
		\small * Studying control designs for high performance electric motors.
		
		\small \textbf{Designing and implementing additional control systems:}\\
		\small * Robust controllers for hydraulic actuators with a focus on precision optimization, applied on embedded controller boards.\\
		\small * Modeling hydraulic actuators and solenoid valves in MATLAB/Simulink.\\
		\small * Developing new methods for estimating the motion states of a car through ego-motion estimation based on real-time video processing and vehicle dynamics on embedded systems.
		
		\item{2014 -- 2017}
		{Università degli studi di Bergamo}
		{PhD researcher}
		
		\small \textbf{Motion state extraction:}\\		
		The motion states of predefined industrial objects were extracted from the fusion of camera data and IMU (accelerometer) in real time to be used as feedback for force control over robot manipulators
		Algorithms were developed in C++ using the Pyramidal Lucas Kanade library from OpenCV and the Boost library
		Parallel methods on both the CPU and GPU were employed to optimize the speed of these processes
		A new sensor fusion method was developed to combine velocity data from a lower resolution camera with higher resolution velocity data obtained through the integration of accelerometer data.
		
		\small \textbf{Shaking table model development:}\\		
		An empirical nonlinear model was developed for a servo-hydraulic uni-axial shaking table based on fluid mechanical expressions and a modified effective bulk modulus model of hydraulic oil.
		The model was able to accurately simulate the acceleration, velocity, and position outputs of the system in response to different types of inputs, such as pulse and sinusoidal signals, across a wide range of frequencies and different specimen weights.
		The model can be used to design and optimize the parameters of a model-based controller for tracking reference force or acceleration signals, which is the goal of the shaking table with only position sensor
		The parameters of the simulated model were estimated using the nonlinear least square method in MATLAB.
		
		\small \textbf{Robot manipulator modeling and control:}\\		
		A toolbox was developed in MATLAB for symbolically modeling the kinematics and dynamics of robot manipulators
		The toolbox was used to model and control a multi-DOF robot manipulator using sliding mode control.
		(\url{https://www.mathworks.com/matlabcentral/fileexchange/50523-advanced-robot-manipulator-simulator})
		
		\small \textbf{Motor inverter driver board design and development:}\\		
		A 300W brushless DC motor inverter driver board was designed and developed for controlling robotic actuators
		The design included PCBs and writing firmwares for STM32 F4XX electronics embedded systems.
		
		\small \textbf{Industrial IoT implementation:}\\		
		Industrial IoT was implemented for various robotic systems based on the AWS IoT platform and Raspberry Pi3.
		
		
		\vspace{-3mm}
		\item{2011 -- 2014}
		{Sedna Fidar Payeh (StartUp)}
		{Research \& Development}
		
		* Working on the sensor and data fusion in the chemical plants.(Developing various technologies based on the realtime video processing).
		
		\item{2008 -- 2011}
		{Electronic Azarbayejan}
		{Research \& Development}
		
		* Research and Design Electronic and Mechanical Systems, Specially Electronic Power Transformation systems.
		
	\end{eventlist}
	\vspace{-10mm}
	\section{Education}
	
	\begin{yearlist}
		
		\item{\small 2014 -- 2017}
		{Ph.D-Eng Mechatronic Engineering}
		{Università degli studi di Bergamo, Italy}
		
		\item{\small 2008 -- 2014}
		{M.Sc-Eng Instrumentation and Automation Engineering}
		{Petroleum University of Technology, Iran}
		
		\item{\small 2003 -- 2008}
		{B.Sc-Eng Electronic Engineering}
		{Urmia University, Iran}
		
	\end{yearlist}
	\vspace{-10mm}
	\section{Freelance Projects}
	
    \begin{itemize}
		\item \small Research and design on the various sensor types for textile yarn breakage based on the image and video processing.

		\item \small Design a walnut’s sorting based on the neural-network image processing in MATLAB.
		\item \small Design and implementation of high accuracy Blood Cell Counter with image processing in MATLAB.	
		\item \small Indirect Flare Temperature estimation from real time video of the flares with Tensorflow and Keras.
		\item \small Design and Implementation of Kalman Filter on AVR microcontrollers(8Bit Processor types)
    \end{itemize}
	\vspace{-10mm}
	\section{Communication skills}
	
	\begin{factlist}
		\item{Persian, Azeri}{Native speaker}
		\item{English}{C1}
		\item{German, Italian}{B1}
	\end{factlist}
	\vspace{-10mm}
	\section{Skills}
	
	\begin{factlist}
		
		\item{Good level}
		{MATLAB/Simulink, Hardware/ Software -in- the-loop, C/C++, Python, OpenCV, Boost library, Keras, Tensorflow, Multisim, Proteus, Altium Designer, Keil uVision, FreeRTOS, CAN, SPI, Solidworks,\LaTeX}
		
		\item{Intermediate}
		{VB, SIEMENS PLC ecosystem, ROS, Ansys, COMSOL, Networks}
		
	\end{factlist}
	%\vspace{-10mm}
%	\section{Awards}
%	\begin{itemize}
%		\item Full Scholarship in M.Scs. from ministry of Petroleum 2008.
%		\item Full Scholarship in Ph.D. from Bergamo University for mechatronics 2014.
%		\item Full funds for all conferences and publication from Bergamo University.
%		\item Full scholarship for summer school on mechanism design for applications MDA 2016 Palermo University, Italy.
%		\item DAAD Full scholarship for summer school on Robot Operating System (ROS) 2017 FH Aachen University, Germany.
%		\item Winning the research grant of university of Bergamo for the project with title "Product and process integration for the realization of electric motors for road vehicles".
%	\end{itemize}
	\vspace{-3mm}
	\section{Awards}
\begin{itemize}
	\item Full Scholarship in M.Scs. from ministry of Petroleum 2008.
	\item Full Scholarship in Ph.D. from Bergamo University for mechatronics 2014.
	\item Full funds for all conferences and publication from Bergamo University.
	\item Full scholarship for summer school on mechanism design for applications MDA 2016 Palermo University, Italy.
	\item DAAD Full scholarship for summer school on Robot Operating System (ROS) 2017 FH Aachen University, Germany.
	\item Winning the research grant of university of Bergamo for the project with title "Product and process integration for the realization of electric motors for road vehicles".
\end{itemize}
\vspace{-10mm}
	\section{Publications}
	\begin{itemize}
		\item Olama, E.K.; Valiloo, S., "A fast wavelet denoising method," IEEE conference in Computer Research and Development (ICCRD), 2011 3rd International Conference on, vol.1, no., pp.492, 494, 11-13 March 2011. Doi:\\10.1109/ICCRD.2011.5764065
		
		\item E. Khadem Olama, H. Jazayeri-Rad, "Online Averaging Wavelet Denoising Method," IEEE conference in Computer Modelling and Simulation, UKSIM European Symposium on, pp. 202-204, 2011 UKSim 5th  European Symposium on Computer Modelling and Simulation, 2011 ISBN: 978-0-7695-4619-3
		
		\item Valiloo, S.; Olama, E.K.; Olama, A.K., "A sliding mode controller with generalized H2 performance for dynamic of nonholonomic mobile robot," IEEE conference in AI \& Robotics and 5th RoboCup Iran Open International Symposium (RIOS), 2013 3rd Joint Conference of , vol., no., pp.1,7, 8-8 April 2013 Doi:\\ 10.1109/RIOS.2013.6595329
		
		\item P. Righettini, R. Strada, E. K. Olama and S. Valilou, "Symbolic kinematic and dynamic modelling toolbox for Multi-DOF robotic manipulators," Automation and Computing (ICAC), 2015 21st International Conference on, Glasgow, 2015, pp. 1-7. doi:\\10.1109/IConAC.2015.7313939
		
		\item P. Righettini, R. Strada, S. Valilou and E. Khadem Olama, "Output feedback sliding mode controller with H2 performance for robot manipulator," Automation and Computing (ICAC), 2015 21st International Conference on, Glasgow, 2015, pp. 1-6. doi:\\10.1109/IConAC.2015.7313943
		
		\item Paolo Righettini, Roberto Strada, Shirin Valilou, Ehsan Khademolama, “Nonlinear Modeling and Experimental Validation of Uni-Axial Servo-Hydraulic Shaking Table”, BATH/ASME 2016 Symposium on Fluid Power and Motion Control Bath, UK, September 7–9, 2016, Paper No. FPMC2016-1773, pp. V001T01A037; 8 pages, doi:\\10.1115/FPMC2016-1773.
		
		\item Paolo Righettini, Roberto Strada, Shirin Valilou, Ehsan Khademolama, “Gray-Box Acceleration Modeling of an Electro Hydraulic Servo Shaking Table with Neural Network”, 2017 IEEE International Conference on Advanced Intelligent Mechatronics, July 3-7, 2017, Sheraton Arabella Park Hotel, Munich, Germany doi:\\10.1109/AIM.2017.8014212.
		
		\item Paolo Righettini, Roberto Strada, Ehsan KhademOlama, Shirin Valilou, Online Wavelet Complementary velocity Estimator, ISA Transactions,2018,ISSN 0019-0578, \url{https://doi.org/10.1016/j.isatra.2017.12.013}. 
		
		\item P. Righettini, R. Strada, S. Valilou, E. Khademolama, "Nonlinear Model of a Servo-Hydraulic Shaking Table with Dynamic Model of Effective Bulk Modulus" Mechanical Systems and Signal Processing, Elsevier,2018, \url{https://doi.org/10.1016/j.ymssp.2018.03.024}.
		
	\end{itemize}
\end{document}