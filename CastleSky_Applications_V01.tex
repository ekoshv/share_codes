\documentclass[12pt]{article}
\usepackage{hyperref}

\title{Applications of Physics Informed Neural Networks}
\author{Ehsan KhademOlama}
\date{\today}

\begin{document}
	
	\maketitle
	\tableofcontents
	
	\section{Introduction}
	\subsection{Background}
	Physics Informed Neural Networks (PINNs) have emerged as a groundbreaking tool for solving complex problems across a multitude of scientific and engineering disciplines \cite{nature-multiscale, springer-multiscale}.
	
	\subsection{Objective}
	The objective of this comprehensive report is to delve into the multifaceted applications of PINNs, with a focus on their transformative potential to democratize high-end technologies \cite{sciencedirect-multiscale}.
	
	\subsection{Scope}
	This report encompasses a wide array of applications, from multiscale physics \cite{arxiv-multiscale} and power systems \cite{ieee-xplore1, ieee-xplore2} to medical imaging and robotics, aiming to provide a holistic view of the capabilities of PINNs \cite{ieee-xplore1}.
		
	\section{Applications of PINNs}
	\subsection{Multiscale Physics}
	Tremendous strides have been made in the realm of multiscale physics, spanning a diverse array of applications from geophysics to biophysics \cite{nature-reviews-physics}. Physics Informed Neural Networks (PINNs) have been particularly instrumental in this regard. They encode model equations like Partial Differential Equations (PDEs) and are adept at solving a variety of equations, including PDEs, fractional equations, integral-differential equations, and even stochastic PDEs \cite{springer-pde}.
	
	Moreover, Graph-Informed Neural Networks (GINNs) have emerged as a specialized subset of PINNs, specifically tailored for multiscale physics applications \cite{sciencedirect-ginns}. A recent deep learning framework has even extended the capabilities of PINNs to multi-scale models, offering promising avenues for future research \cite{arxiv-multiscale-models}.
		
	\subsection{Power Systems}
	Physics-Informed Neural Networks (PINNs) have found significant applications in the domain of power systems. They are employed for a multitude of tasks including state and parameter estimation, dynamic analysis, power flow calculation, and optimal power flow \cite{ieee-xplore1}.
	
	Furthermore, a novel framework has been introduced that specifically tailors PINNs for power system applications, thereby opening new avenues for research and development \cite{ieee-xplore2}.
	
	There is a growing consensus in the scientific community that PINNs can effectively address various concerns in power systems. This is achieved by seamlessly integrating physics-informed rules or laws into state-of-the-art deep learning algorithms \cite{ieee-xplore1}.
	
	
	
	\subsection{Optical Fiber Communications}
	Physics-Informed Neural Networks (PINNs) have been increasingly applied in the field of optical fiber communications. They have been studied to solve the nonlinear Schrödinger equation, which is crucial for understanding nonlinear dynamics in fiber optics \cite{researchgate-optical}.
	
	A systematic investigation and comprehensive verification on the use of PINNs for multiple physical effects in optical fibers have been carried out. This includes solving complex equations that govern the behavior of light within the fiber \cite{wiley-optical}.
	
	Furthermore, PINNs have been constructed to solve the nonlinear Schrödinger equation for different input waveforms, thereby enhancing the performance and reliability of optical fiber communication systems \cite{ieee-xplore-optical}.
		
	\subsection{Medical Imaging}
	Physics-Informed Neural Networks (PINNs) have been increasingly applied in the field of medical imaging. They are used for direct imaging tasks, offering a crucial inversion-based approach that is particularly useful in medical applications \cite{ieee-medical}.
	
	Additionally, PINNs have been leveraged in a deep learning approach for near-field microscopy, which has significant implications for medical imaging. This approach allows for the inversion of complex optical parameters in nanostructured environments \cite{aip-medical}.
	
	Moreover, physics-informed deep learning frameworks are being used to augment sparse clinical measurements, thereby generating physically consistent brain hemodynamic parameters with high spatiotemporal resolution \cite{ncbi-medical}.
	
	\subsection{Boundary Layer Problems}
	Boundary layer problems have implications in real-time problems across various industries, including aerospace, automotive, and HVAC systems. Physics-Informed Neural Networks (PINNs) have been applied to these problems, offering innovative solutions. Theory-guided PINNs have been used to address boundary layer problems with singular perturbation \cite{sciencedirect-boundary1}.
	
	Moreover, PINNs have been employed to solve thermal-fluid problems in boundary layers, which is particularly relevant for HVAC systems \cite{sciencedirect-boundary2}. The key component in every PINN when solving these problems is the boundary conditions, which constrain the underlying boundary value problem \cite{arxiv-boundary}.
	
	The application of PINNs in solving boundary layer problems could lead to more efficient and cost-effective solutions in aerospace, automotive, and HVAC systems, aligning well with the goal of making high-end technologies more affordable and impactful.
	
	\subsection{Real-time HVAC Systems}
	Using Physics-Informed Neural Networks (PINNs) in real-time HVAC systems could be groundbreaking. Imagine a smart HVAC system that not only adjusts the temperature but also understands the thermal dynamics of the entire building. It could predict how different rooms will heat up or cool down throughout the day and adjust the HVAC settings accordingly. This would not only make the system more efficient but also reduce energy consumption, thereby making it more cost-effective. Such a system could be particularly useful in large commercial buildings, hospitals, or even in smart cities, aligning well with the goal of making high-end technologies more affordable and impactful \cite{nature-hvac2}.
	
	Recent research indicates that solving real-time control problems in HVAC systems may still be intractable even when solving is possible in real-time \cite{nature-hvac1}. However, PINNs leverage deep learning methods to provide an alternative approach to real-time modeling, offering a promising avenue for future developments \cite{nature-hvac2}.
		
	\subsection{Electromechanical Systems}
	While there are limited direct references to the application of Physics-Informed Neural Networks (PINNs) in electromechanical systems, their potential utility in this domain is evident. PINNs are particularly effective for solving ill-posed and inverse problems, which are commonly encountered in electromechanical systems \cite{nature-electromechanical}. 
	
	Moreover, PINNs have been implemented to investigate mixed electroosmotic pressure-driven flow in microchannels. This has potential applications in electromechanical systems, particularly in the design and optimization of microfluidic devices \cite{sciencedirect-electromechanical}.
		
	\subsection{Hydraulic Systems}
	Physics-Informed Neural Networks (PINNs) have shown promise in the field of hydraulic systems, particularly in pipeline systems. They are employed for hydraulic transient analysis, aiding in the monitoring and prediction of hydraulic transient events. This ensures the proper operation of pressure control devices in hydraulic systems \cite{sciencedirect-hydraulic}. 
	
	Furthermore, PINNs have been applied to hydrodynamic voltammetry, a technique commonly used in hydraulic systems. This application showcases the versatility of PINNs in addressing complex hydraulic problems \cite{researchgate-hydraulic}.
	
	\subsection{Pneumatic Systems}
	Physics-Informed Neural Networks (PINNs) have been gaining attention in the field of pneumatic systems, particularly in soft pneumatic actuators. The use of Physics-Informed Recurrent Neural Networks (PIRNNs) has been demonstrated in a hybrid prediction scheme on two typical soft pneumatic actuators: a McKibben pneumatic artificial muscle and a pneumatic-based soft finger made of silicone \cite{ieee-pneumatic}. 
	
	This approach has garnered significant attention recently and is considered a promising avenue for further research and development in pneumatic systems \cite{researchgate-pneumatic}.
		
	\subsection{Sensor Fusion with PINN}
	Physics-Informed Neural Networks (PINNs) have been increasingly applied in the field of sensor fusion. A multi-fidelity physics informed deep neural network (MF-PIDNN) has been proposed that blends both physics-informed and data-driven deep neural networks. This approach is particularly suitable when the governing equation of the sensors is approximately known \cite{sciencedirect-sensor}. 
	
	Additionally, PINNs have been used to solve a variety of equations, including PDEs, fractional equations, integral-differential equations, and stochastic PDEs. This makes them a versatile tool in sensor fusion applications, where multiple types of data and equations may be involved \cite{springer-sensor}.
		
	\subsection{Robotics}
	Physics-Informed Neural Networks (PINNs) have been increasingly applied in the field of robotics, specifically in modeling and control systems. They offer both theoretical and experimental advantages, enabling more precise and efficient control of robotic mechanisms \cite{researchgate-robotics1}. 
	
	Additionally, the application of physics-informed neural operators in 2D incompressible magnetohydrodynamics simulations has implications for robotic applications. These operators can be used to model complex physical interactions in robotic systems, thereby enhancing their performance and reliability \cite{researchgate-robotics2}.
		
	\section{Challenges and Future Work}
	While PINNs offer numerous advantages, they also present challenges such as computational complexity, data availability, model accuracy, and hardware limitations. Future work will focus on overcoming these hurdles to unlock their full potential.
	
	\section{Conclusion}
	The diverse applications of PINNs, as explored in this report, hold immense promise for making high-end technologies more affordable and accessible, thereby having a far-reaching impact on society.
	
	\section{References}
	\begin{thebibliography}{9}
		
		\bibitem{nature-multiscale}
		Nature Reviews Physics,
		\textit{Tremendous strides in multiscale physics},
		\url{https://www.nature.com/articles/s42254-021-00314-5}.
		
		\bibitem{springer-multiscale}
		Springer,
		\textit{PINNs in Multiscale Physics},
		\url{https://link.springer.com/article/10.1007/s10915-022-01939-z}.
		
		\bibitem{sciencedirect-multiscale}
		ScienceDirect, 
		\textit{GINNs: Graph-Informed Neural Networks for multiscale physics}, \url{https://www.sciencedirect.com/science/article/pii/S0021999121000875}
		
		\bibitem{arxiv-multiscale}
		arXiv,
		\textit{A deep learning framework for multi-scale models based on physics-informed neural networks}, \url{https://arxiv.org/abs/2308.06672}

		\bibitem{ieee-xplore1}
		IEEE Xplore,
		\textit{Applications of Physics-Informed Neural Networks in Power Systems - A Review}, \url{https://ieeexplore.ieee.org/abstract/document/9743327}
		
		\bibitem{ieee-xplore2}
		IEEE Xplore,
		\textit{Physics-Informed Neural Networks for Power Systems}, \url{https://ieeexplore.ieee.org/document/9282004}
		
		\bibitem{nature-reviews-physics}
		Nature Reviews Physics,
		\textit{Physics-informed machine learning}, \url{https://www.nature.com/articles/s42254-021-00314-5}.
		
		\bibitem{springer-pde}
		Springer,
		\textit{Scientific Machine Learning Through Physics–Informed Neural Networks: Where we are and What’s Next}, \url{https://link.springer.com/article/10.1007/s10915-022-01939-z}.
		
		\bibitem{sciencedirect-ginns}
		ScienceDirect,
		\textit{GINNs: Graph-Informed Neural Networks for multiscale physics}, \url{https://www.sciencedirect.com/science/article/pii/S0021999121000875}.
		
		\bibitem{arxiv-multiscale-models}
		arXiv,
		\textit{A deep learning framework for multi-scale models based on physics-informed neural networks}, \url{https://arxiv.org/abs/2308.06672}.
				
		\bibitem{researchgate-optical}
		ResearchGate,
		\textit{Applications of Physics-Informed Neural Network for Optical Fiber Communications},
		\url{https://www.researchgate.net/publication/361772364_Applications_of_Physics-Informed_Neural_Network_for_Optical_Fiber_Communications}.
		
		\bibitem{wiley-optical}
		Wiley Online Library,
		\textit{Systematic Investigation of PINNs in Optical Fibers},
		\url{https://onlinelibrary.wiley.com/doi/10.1002/lpor.202100483}.
		
		\bibitem{ieee-xplore-optical}
		IEEE Xplore,
		\textit{Enhancing Optical Fiber Communication Systems with PINNs},
		\url{https://ieeexplore.ieee.org/document/9489953/}.
		
		\bibitem{ieee-medical}
		IEEE Xplore,
		\textit{Direct Imaging Tasks in Medical Applications with PINNs},
		\url{https://ieeexplore.ieee.org/document/9897269}.
		
		\bibitem{aip-medical}
		AIP,
		\textit{Deep Learning Approach for Near-Field Microscopy in Medical Imaging},
		\url{https://pubs.aip.org/aip/app/article/7/1/010802/2835099}.
		
		\bibitem{ncbi-medical}
		NCBI,
		\textit{Augmenting Sparse Clinical Measurements with Physics-Informed Deep Learning},
		\url{https://www.ncbi.nlm.nih.gov/pmc/articles/PMC9437127}.
		
		\bibitem{sciencedirect-boundary1}
		ScienceDirect,
		\textit{Theory-guided PINNs for Boundary Layer Problems with Singular Perturbation},
		\url{https://www.sciencedirect.com/science/article/pii/S0021999122008312}.
		
		\bibitem{sciencedirect-boundary2}
		ScienceDirect,
		\textit{PINNs for Thermal-Fluid Problems in Boundary Layers},
		\url{https://www.sciencedirect.com/science/article/pii/S0735193322000124}.
		
		\bibitem{arxiv-boundary}
		arXiv,
		\textit{Boundary Conditions in PINNs for Boundary Layer Problems},
		\url{https://arxiv.org/abs/2310.02548}.
		
		\bibitem{nature-hvac2}
		Nature,
		\textit{PINNs for Real-Time Modeling in HVAC Systems},
		\url{https://www.nature.com/articles/s41598-023-43325-1}.
		
		\bibitem{nature-hvac1}
		Nature,
		\textit{Intractability of Real-Time Control Problems in HVAC Systems},
		\url{https://www.nature.com/articles/s41598-023-36799-6}.
		
		\bibitem{nature-electromechanical}
		Nature Reviews Physics,
		\textit{Ill-Posed and Inverse Problems in Electromechanical Systems},
		\url{https://www.nature.com/articles/s42254-021-00314-5}.
		
		\bibitem{sciencedirect-electromechanical}
		ScienceDirect,
		\textit{Mixed Electroosmotic Pressure-Driven Flow in Microchannels},
		\url{https://www.sciencedirect.com/science/article/pii/S0255270123002775}.
		
		\bibitem{sciencedirect-hydraulic}
		ScienceDirect,
		\textit{Hydraulic Transient Analysis in Pipeline Systems},
		\url{https://www.sciencedirect.com/science/article/pii/S0043135422007771}.
		
		\bibitem{researchgate-hydraulic}
		ResearchGate,
		\textit{The Application of Physics-Informed Neural Networks to Hydrodynamic Voltammetry},
		\url{https://www.researchgate.net/publication/359856117_The_Application_of_Physics-Informed_Neural_Networks_to_Hydrodynamic_Voltammetry}.
		
		\bibitem{ieee-pneumatic}
		IEEE Xplore,
		\textit{Hybrid Prediction Scheme on Soft Pneumatic Actuators},
		\url{https://ieeexplore.ieee.org/document/9783081}.
		
		\bibitem{researchgate-pneumatic}
		ResearchGate,
		\textit{Physics-Informed Recurrent Neural Networks for Soft Pneumatic Actuators},
		\url{https://www.researchgate.net/publication/360907965_Physics-Informed_Recurrent_Neural_Networks_for_Soft_Pneumatic_Actuators}.
		
		\bibitem{sciencedirect-sensor}
		ScienceDirect,
		\textit{Multi-Fidelity Physics Informed Deep Neural Network for Sensor Fusion},
		\url{https://www.sciencedirect.com/science/article/pii/S0021999120307166}.
		
		\bibitem{springer-sensor}
		Springer,
		\textit{Versatile Equations Solving with PINNs in Sensor Fusion},
		\url{https://link.springer.com/article/10.1007/s10915-022-01939-z}.
		
		\bibitem{researchgate-robotics1}
		ResearchGate,
		\textit{Physics-informed Neural Networks to Model and Control Robots: A Theoretical and Experimental Investigation},
		\url{https://www.researchgate.net/publication/370633879_Physics-informed_Neural_Networks_to_Model_and_Control_Robots_a_Theoretical_and_Experimental_Investigation}.
		
		\bibitem{researchgate-robotics2}
		ResearchGate,
		\textit{Applications of Physics-Informed Neural Operators in 2D Incompressible Magnetohydrodynamics Simulations},
		\url{https://www.researchgate.net/publication/370396402_Applications_of_physics_informed_neural_operators}.
		
		
	\end{thebibliography}

\end{document}
